\chapter*{}
%\thispagestyle{empty}
%\cleardoublepage

%\thispagestyle{empty}

\input{portada/portada_2}



\cleardoublepage
\thispagestyle{empty}

\begin{center}
{\large\bfseries Análisis de los algoritmos de Boosting en entornos con ruido de clase.}\\
\end{center}
\begin{center}
Rafael Nogales Vaquero \\
\end{center}

%\vspace{0.7cm}
\noindent{\textbf{Palabras clave}: Machine Learning, Boosting, Ruido de clase, Función de pérdida, Problemas de Clasificación}\\

\vspace{0.7cm}
\noindent{\textbf{Resumen}}\\

En este TFG se explican los motivos por los cuales los algoritmos de boosting son especialmente sensibles al ruido de clase y se comentan las formas en las que se puede paliar en la medida de lo posible esta deficiencia.\cleardoublepage


\thispagestyle{empty}


\begin{center}
{\large\bfseries Analysis of Boosting algorithms in environments with class noise.}\\
\end{center}
\begin{center}
Rafael Nogales Vaquero\\
\end{center}

%\vspace{0.7cm}
\noindent{\textbf{Keywords}: Machine Learning, Boosting, Class noise, Loss function, Classification problems}\\

\vspace{0.7cm}
\noindent{\textbf{Abstract}}\\
In this TFG we explain the reasons why boosting algorithms are especially sensitive to class noise and we discuss the ways in which this deficiency can be alleviated as far as possible.

\chapter*{}
\thispagestyle{empty}

\noindent\rule[-1ex]{\textwidth}{2pt}\\[4.5ex]

Yo, \textbf{Rafael Nogales Vaquero}, alumno de la titulación DOBLE GRADO EN INGENIERÍA INFORMÁTICA Y MATEMÁTICAS de la \textbf{Escuela Técnica Superior
de Ingenierías Informática y de Telecomunicación de la Universidad de Granada}, con DNI 76669550Q, autorizo la
ubicación de la siguiente copia de mi Trabajo Fin de Grado en la biblioteca del centro para que pueda ser
consultada por las personas que lo deseen.

\vspace{6cm}

\noindent Fdo: Rafael Nogales Vaquero

\vspace{2cm}

\begin{flushright}
Granada a 7 de Septiembre de 2018
\end{flushright}


\chapter*{}
\thispagestyle{empty}

\noindent\rule[-1ex]{\textwidth}{2pt}\\[4.5ex]

D. \textbf{Francisco Herrera Triguero  }, Profesor del Departamento de Ciencias de la Computación e Inteligencia Artificial  de la Universidad de Granada.

\vspace{0.5cm}

D. \textbf{Julian Luengo Martín}, Profesor del Departamento de Ciencias de la Computación e Inteligencia Artificial  de la Universidad de Granada.


\vspace{0.5cm}

\textbf{Informan:}

\vspace{0.5cm}

Que el presente trabajo, titulado \textit{\textbf{Análisis de los algoritmos de Boosting en entornos con ruido de clase}},
ha sido realizado bajo su supervisión por \textbf{Rafael Nogales Vaquero}, y autorizamos la defensa de dicho trabajo ante el tribunal
que corresponda.

\vspace{0.5cm}

Y para que conste, expiden y firman el presente informe en Granada a 7 de Septiembre de 2018 .

\vspace{1cm}

\textbf{Los directores:}

\vspace{5cm}

\noindent \textbf{Francisco Herrera Triguero   \ \ \ \ \ Julian Luengo Martín }

\chapter*{Agradecimientos}
\thispagestyle{empty}

       \vspace{1cm}


A mi familia, amigos y profesores

