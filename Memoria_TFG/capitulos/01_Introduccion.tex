\chapter{Introducción}

\section{Contexto}

El objetivo final de este trabajo es comprender y desarrollar clasificadores de tipo boosting utilizando los mecanismos por los cuales el algoritmo XGBoost (el algoritmo de Boosting más utilizado a nivel global) es más robusto al ruido de clase.  

Este algoritmo es utilizado en problemas de ambitos muy diferentes siempre que haya que trabajar con datos tabulados suele dar un gran resultado.  
Por desgracia hay muchos problemas con datos tabulados que son susceptibles de tener datos mal clasificados y esto es un gran problema para los algoritmos de boosting clásicos como Adaboost.  

El problema principal radica en que para ajustarse rápidamente a los datos los algoritmos de boosting suelen utilizar funciones de pérdida exponenciales que dan una importancia enorme a los datos ruidosos. 

Lo cual nos deja un amplio terreno sobre el que trabajar.

\section{Solución}
Debido a que el nuestro mayor problema es el sobreajuste debido a la función de pérdida exponencial ya que da un peso cada vez mayor a los datos difíciles de clasificar y los ruidosos siempre lo son.  
Es por esto que la estrategia a seguir para mitigar este problema va a consistir en utilizar otras funciones de pérdida más robustas al ruido.  
Previamente repasaremos todos los conceptos matemáticos relacionados con el boosting, las funciones de pérdida y el aprendizaje automático en general. 

\section{Herramientas}
Clasificaremos nuestras herramientas en dos tipos: Herramientas matemáticas y herramientas software.  

Para la parte matemática nuestras herramientas serán el cálculo diferencial, la estadística y el álgebra lineal.  

En la parte informática utilizaremos principalmente R, Python y algunas de sus librerías: pandas, scikit-learn, xgboost, matplotlib...

\section{Fuentes}

http://zhanpengfang.github.io/418home.html
https://www.analyticsvidhya.com/blog/2016/03/complete-guide-parameter-tuning-xgboost-with-codes-python/
http://www.aprendemachinelearning.com/que-es-overfitting-y-underfitting-y-como-solucionarlo/
