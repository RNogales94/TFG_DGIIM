\chapter{Objetivos}

\section{Estudio general de los algoritmos de boosting}
En este trabajo se realiza un estudio de la complejidad del problema de clasificación en el ámbito del aprendizaje automático. En particular, se analiza la dificultad de clasificación en los problemas donde aplicamos algoritmos de boosting en presencia de ruido de clase.

\section{Analizar cómo el ruido les afecta}
Se explican así mismo cuales son los principales defectos de los algoritmos de boosting cuando tratamos con datos ruidosos: El modelo cae en \textit{over-fitting} o bien si aplicamos las técnicas de \textit{regularización} de forma demasiado estricta caemos en \textit{under-fitting}


\section{Proponer una mejora para hacerlos más robustos al ruido}
En el trabajo se explican tres formas de conseguir que un modelo sea más robusto al ruido:
\begin{itemize}
\item{Incorporación de filtros de ruido}
\item{Función de pérdida robusta}
\item{Aplicar técnicas de regularización}

\end{itemize}